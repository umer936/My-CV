%%%%%%%%%%%%%%%%%%%%%%%%%%%%%%%%%%%%%%%%%
% Umer936 Resume
% XeLaTeX
%
% Original author:
% Adrien Friggeri (adrien@friggeri.net)
% https://github.com/afriggeri/
%
% CV of:
% Umer Salman (umer936@gmail.com)
% https://github.com/umer936/My-CV
%
% License:
% CC BY-NC-SA 3.0 (http://creativecommons.org/licenses/by-nc-sa/3.0/)
%
%%%%%%%%%%%%%%%%%%%%%%%%%%%%%%%%%%%%%%%%%

\documentclass[print]{friggeri-cv} % Add 'print' as an option into the square bracket to remove colors for printing
\usepackage[super]{nth}
\usepackage{lastpage}
\usepackage{fancyhdr}
\usepackage{units}
\long\def\/*#1*/{}
\pagestyle{fancy}
\fancyhf{} % to clear existing header/footer if you don't want it
\renewcommand\headrulewidth{0pt}

\begin{document}
\header{umer}{salman}{\noindent\rule{10cm}{0.4pt}} % Your name and current job title/field
% \rfoot{\thepage\ of \pageref{LastPage}}

%----------------------------------------------------------------------------------------
%	SIDEBAR SECTION
%----------------------------------------------------------------------------------------

\begin{aside} % In the aside, each new line forces a line break
	\section{contact}
	24 Galleria Drive
	San Antonio, TX 78257
	USA
	~
	(936) 463 8626
	~
	\href{mailto:umer936@gmail.com}{umer936@gmail.com}
	\href{mailto:umer936@utexas.edu}{umer936@utexas.edu}
	\href{http://umer936.com}{umer936.com}
	~
	\href{http://github.com/umer936}{github.com/umer936}
	\href{http://facebook.com/Umer936}{fb://Umer936}
	~
	\section{programming}
	CSS3 \& HTML5
	PHP ({\color{red} $\varheartsuit$} Laravel)
	MySQL
	JavaScript
	jQuery
	Java, C, C++
	\LaTeX \ (this r\'esum\'e)
	Python
	TIBASIC
	Assembly
	MATLAB
	LabVIEW
	Git, SVN, Mercurial
	~
	\section{operating systems}
	Linux
	Android
	OS X
	iOS
	Windows
	~
	\section{software skills}
	SolidWorks
	Burp Suite
	ROS2
	Eclipse
	OpenCV
	Android Studio
	ADB/Fastboot
\end{aside}

%----------------------------------------------------------------------------------------
%	EDUCATION SECTION
%----------------------------------------------------------------------------------------

\section{education}
\vspace{-10pt}

\begin{entrylist}

	\entry
	{2016--Now}
	{Electrical and Computer Engineering - Fall 2020}
	{The University of Texas at Austin}
	{
		Tech Cores: Software Engineering \& Design and Academic Enrichment \\ 
		Minor: Business \\
		\textbf{Embedded Systems Lab Keyboard:} \\ 
			Created a USB 10-Keyless RGB Keyboard with Macro Recording using TM4C microcontroller \href{https://youtu.be/fSX6dDx9jvY}{https://youtu.be/fSX6dDx9jvY} \\
		\textbf{Entrepreneurship Senior Design (CTO):} 
		\begin{itemize}
			\item Built a hardware/software platform to process physiological indicators in children
			\begin{itemize}
				\item Designed PCB with galvanic skin response sensor, heart rate sensor, and microphone for wearable
				\item Built Azure/Tableau web platform for healthcare professionals to view
			\end{itemize}
			\item Formed a business plan and presented to biomedical industry experts 
		\end{itemize}
	}
	
\end{entrylist}

\vspace{-15pt}
\section{extracurriculars}
\vspace{-10pt}

\begin{entrylist}

	\entry
	{2017--2020}
	{Texas Aerial Robotics (TAR)}
	{The University of Texas at Austin}
	{
	\emph{Founder and President}
		\begin{itemize}
			\item Lead a 40 person org. to compete in the International Aerial Robotics Competition
			\item Build and program cutting-edge, fully autonomous quadcopters
			\item Use computer vision and sensors, such as LiDAR and optical flow, to target and interact with moving ground robots in a GPS-denied environment
			\item Research drone swarming and drone control through human voice and gestures
			\item Develop abilities to interact with modules on moving reference frames (boats) 2mi away
		\end{itemize}
	}

\end{entrylist}

%----------------------------------------------------------------------------------------
%	WORK EXPERIENCE SECTION
%----------------------------------------------------------------------------------------

\vspace{-15pt}
\section{experience}
\vspace{-10pt}

\begin{entrylist}

	\entry
	{2020}
	{National Geospacial-Intelligence Agency (WFH)}
	{San Antonio, Texas}
	{\emph{Cybersecurity/Software Development Intern}
		\begin{itemize}
			\item Developed a grammar assistance tool in JavaScript for Weekly Activity Reports
			\begin{itemize}
				\item Corrects ~6,000 writing mistakes to reduce the amount of time in editing process 
				\item Planned for use across NGA, such as in reports to Congress, etc 
			\end{itemize}
			\item Assisted in cyber-deception (honeypot ``network devices'') tool analysis
		\end{itemize}
	}
\vspace{-5pt}

	%-----------------------------------------------
	
	\entry
	{2019}
	{National Geospacial-Intelligence Agency}
	{St. Louis, Missouri}
	{\emph{Cybersecurity Intern}
		\begin{itemize}
			\item Active Clearance: TS/SCI
			\item Assisted Detect and Incident Response teams
			\item Improved rulesets for Linux compromise detection
			\item Built PowerShell scripts to improve threat detection and analysis
		\end{itemize}
	}
\vspace{-5pt}
	
	%-----------------------------------------------
	
	\entry
	{2018}
	{Visa Inc.}
	{Austin, Texas}
	{\emph{Security Engineering Intern}
		\begin{itemize}
			\item Created automated penetration testing suite using Burp Suite and Python to increase security while decreasing time in security testing stage
			\begin{itemize}
				\item Tool tests Visa products \& APIs for vulnerabilities such as XSS and Clickjacking
				\item Learns from cybersecurity team to eliminate false positives
			\end{itemize}
		\end{itemize}
	}
\vspace{-5pt}
	
	%-----------------------------------------------

	\entry
	{2017}
	{General Dynamics Mission Systems}
	{San Antonio, Texas}
	{\emph{Cybersecurity Intern}
		\begin{itemize}
			\item Developed testing tools using OpenCV and Python for PitBull, a Multilevel Security (MLS) Linux OS based on RHEL, and Trusted Network Environment (TNE) 
			%\begin{itemize}
			%	\item PitBull/TNE are used by U.S. Federal Agencies
			%\end{itemize}
			\item Created PitBull Health Console for diagnosing system/network problems
		\end{itemize}
	}
\vspace{-5pt}

	%-----------------------------------------------

	\entry
	{2015}
	{Parlevel Systems}
	{San Antonio, Texas}
	{\emph{Software Development Intern}
		\begin{itemize}
			\item Worked on the development of analytical software for vending machine operators
			\item Developed the company's foray into micromarkets, which became a core product
		\end{itemize}
	}
\vspace{-25pt}

\end{entrylist}

%----------------------------------------------------------------------------------------
%	PROJECTS SECTION
%----------------------------------------------------------------------------------------
%
%\vspace{-14pt}
%\section{team projects}
%\vspace{-8pt}

%\begin{entrylist}
%\end{entrylist}

\end{document}
